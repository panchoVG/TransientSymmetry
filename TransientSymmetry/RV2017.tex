\documentclass[12pt,oneside,a4paper]{article} % for sharing
\usepackage{apacite}
\usepackage{appendix}
\usepackage{amsmath}
\usepackage{amsthm}

\usepackage{amssymb} % for approx greater than
\usepackage{caption}
\usepackage{placeins} % for \FloatBarrier
\usepackage{graphicx}
\usepackage{subcaption}
\usepackage{longtable}
\usepackage{setspace}
\usepackage{booktabs}
\usepackage{tabularx}
\usepackage{xcolor,colortbl}
\usepackage{chngpage}
\usepackage{natbib}
\bibpunct{(}{)}{,}{a}{}{;} 
\usepackage{url}
\usepackage{nth}
\usepackage{authblk}
\usepackage[most]{tcolorbox}
\usepackage[normalem]{ulem}
\usepackage{amsfonts}
% columns for longtable
\newcolumntype{C}[1]{>{\centering\let\newline\\\arraybackslash\hspace{0pt}}m{#1}}
\newcolumntype{L}[1]{>{\raggedright\let\newline\\\arraybackslash\hspace{0pt}}m{#1}}
\usepackage{arydshln} % Dashed lines in matrices

\usepackage[margin=1in]{geometry}
%\doublespacing % for review

% line numbers to make review easier
%\usepackage{lineno}
%\linenumbers

%\usepackage{soul}% for \st{}

%%%%%%%%%%%%%%%%%%%%%%%%%%%%%%%%%%%%%%%%%%%%%%%%%%%%%%%%%%%%%%%%%%%%%%%%%%%%%%
% for section 4 math environments
\theoremstyle{definition}
\newtheorem{definition}{Definition}[section]
\newtheorem{theorem}{Theorem}[section]
\newtheorem{proposition}{Proposition}[section]
\newtheorem{corollary}{Corollary}[proposition]
\newtheorem{remark}{Remark}[section]

%%%%%%%%%%%%%%%%%%%%%%%%%%%%%%%%%%%%%%%%%%%%%%%%%%%%%%%%%%%%%%%%%%%%%%%%%%%%%%

\newcommand\ackn[1]{%
  \begingroup
  \renewcommand\thefootnote{}\footnote{#1}%
  \addtocounter{footnote}{-1}%
  \endgroup
}

% Affiliations in small font size
\renewcommand\Affilfont{\small}

\defcitealias{HMD}{HMD 2016}

% junk for longtable caption
\AtBeginEnvironment{longtable}{\linespread{1}\selectfont}
\setlength{\LTcapwidth}{\linewidth}

% sort van Raalte properly
% #1: sorting key, #2: prefix for citation, #3: prefix for bibliography
\DeclareRobustCommand{\VAN}[3]{#2} % set up for citation

%%%%%%%%%%%%%%%%%%%%%%%%%%%%%%%
\begin{document}

\title{Symmetry in the forward and backward tenure of transient states in
stationary populations}
%\author{author(s) redacted}
\author[1]{Tim Riffe\thanks{riffe@demogr.mpg.de}}
\author[2,3]{Francisco Villavicencio}
\affil[1]{Max Planck Institute for Demographic Research, Rostock, Germany}
\affil[2]{Max-Planck Odense Center on the Biodemography of Aging, Odense, Denmark}
\affil[3]{Department of Public Health, University of Southern Denmark, Odense, Denmark}

\maketitle

\begin{abstract}

\end{abstract}

\section{A proof of transient symmetry}

\begin{theorem}
Given a stationary population and fixed transition rates, the probability that a
randomly selected individual is in state  $s$ and entered the state $x$ years
ago is equal to the probability of being in state $s$ and exiting in $x$ years.
\end{theorem}

\begin{proof}

\begin{enumerate}
\item{} A duration can be represented as a line segment, potentially a
subset of a life-line. Points along a single within-person duration could be continuously
sampled over time, used to bisect the segment, collecting two infinite sets of
values: time spent in the duration and time until exiting the duration. If
continuously collected, these two sets will have the same values, with time
spent in ascending order and time left in descending order. If sorted
identically these two sets will therefore be identical. This is so by way of
complements.

\item{} If this individual is reborn in each instant, destined to relive the
exact same life course as the first, we end up with an infinite
number of identically aligned identically long segments placed side-by-side.
This infinite set of segments in effect forms a plane. One could in this
setting take a census at a single point in time, collecting an infinite set of
time spent and time left values, each from a unique cohort in sequence. It is
clear that the two sets observed at a single time point will be identical
to the first two sets that were observed of a single duration over its entire
length.

\item{} Assume we have another individual from the same birth cohort that enters
the same state as the first, but at a different time and for a different
duration. We could do (1) in the same way, sampling at an infinite set of points
along the duration, and constructing two sets, one of time spent and another of
time left in the state. These too would be identical sets, but they would not be
identical to those of the first individual, as the range of values would be
different. The union of the first and second time-spent sets and the union of
the first and second time-left sets are guaranteed to be identical. 

\item{} If the second individual repeats over time as in (2), then our
comprehensive sample at a single point in time also yields identical sets
of time spent and time left values. Also from this census, the union of the
first and second time-spent sets and the union of the first and second time-left sets
are guaranteed to be identical. 

\item{} By induction we can keep adding durations in this state, infinitely if we
please, and the union of all resultant time spent sets and the union of time
left sets will continue being identical. It is therefore true that the
probability of selecting a particular value from the time-spent set is identical
to the probability of selecting the same value from the time-left set. This
constitutes a proof of symmetry of time spent and left in transient states in
stationary populations.
\end{enumerate}
\end{proof}

\end{document}
